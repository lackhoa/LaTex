\documentclass{article}

%packages go here!
\usepackage{anyfontsize}
\usepackage{tikz} %For drawing the cover page's background
\usepackage{graphicx}
\usepackage{titlesec}
\usepackage[a4paper, 
	right=2cm,
	left=4.3cm,
	top=2.25cm, bottom=1.25cm]{geometry}
\usepackage[document]{ragged2e} %Left alignment
\usepackage{parskip} %Add an additional line after paragraphs
\usepackage{fancyhdr} %Put numbering on top
\usepackage{enumitem} %List handling

%Setting the font
\renewcommand{\rmdefault}{phv} % Arial
\renewcommand{\sfdefault}{phv} % Arial

%Enter parameters here
\newcommand\myauthor{Vo Dang Khoa(T5616SN)}
\newcommand\mytitle{An Arbitrary tilte goes here!}
\newcommand{\subtitle}{Some subtitle goes here!}
\newcommand{\rpas}{Report or Assignment}
\newcommand{\course}{Course name}

\linespread{1.5}

%Redefining maketitle
\renewcommand{\maketitle}{
\thispagestyle{empty}

\begin{center}

\fontsize{16}{19} \selectfont \myauthor

\vspace{20pt}

\MakeUppercase{\fontsize{24}{30} \selectfont \mytitle}

\vspace{5pt}

\fontsize{20}{25} \selectfont \subtitle

\vspace{20pt}

\fontsize{16}{19} \selectfont {\rpas \\ \course}

\vspace{20pt}

\the\year

%Adding the logo
\vspace{100pt}
\includegraphics{logo.jpg}

\end{center}

%Making the cover background
\tikz[remember picture,overlay] \node[inner sep=0pt] at (current page.center){
\includegraphics[width=\paperwidth,height=\paperheight]{coverbg.png}};
\clearpage
}

%Formatting The sections:
\titleformat{\section}
{\bfseries}
{\thesection}
{.17in}
{\MakeUppercase}

%The subsection:
\titleformat{\subsection}
{\bfseries}
{\thesubsection}
{.17in}
{}

%Formatting the paragraphs
%Put new line between paragraphs
\setlength{\parskip}{\baselineskip}
%No indentation
\setlength{\parindent}{0pt}

%Content of the document
\begin{document}
%The cover page
%First we must clear up the margin
\newgeometry{
	right=2cm,
	left=2cm,
	top=4cm,
	bottom=2cm,
	}
{\maketitle}

%Table of content:
\thispagestyle{empty}
{\tableofcontents}

{\clearpage} %% old habits die hard ;-)

%Formatting the margin (If 'restoregeometry' doesn't work then I have no choices":
\newgeometry{
	right=2cm,
	left=4.3cm,
	top=2.25cm,
	bottom=1.25cm,
	}

%Put numbering on top (i.e. the header)
\pagestyle{fancy}
\fancyhf{}
\chead{\thepage}
\renewcommand{\headrulewidth}{0pt} %No line please!

%Formatting lists:
\setlist[itemize]{leftmargin=.6in, nosep} %Vertical spacing of list paragraphs is none

%===========================================================================================================

%CONTENT GOES HERE!
\section{Introduction}
This template for writing reports or assignments at Xamk was last updated on 16 August 2017. The template has been optimised for full, downloadable version of Microsoft Word – not for wordprocessing applications such as Office 365 operating in the cloud.

\section{Layout of reports of assignments}
This chapter introduces the layout of text produced with the template. Appendix 1 
provides more information on the technical details of this template.

\subsection{Title page, table of contents, margins, indentation, fonts, font sizes 
and page numbers}

The layout of the title page, table of contents and list of references do not require students’ own work. The example information of the template can simply be replaced by the details of each report or assignment. Automatic hyphenation can be disabled in the title of the title page.

The title page follows the layout of the title page in this document. The title page includes the student’s own name, the title of the report or assignment, the type of the document − for example practical training report or learning diary − and the date of the report or assignment. There is no page number on the title page, and reports or assignments 
do not include abstracts and forewords or prefaces. 

The heading of the table of contents is CONTENTS, and page 2 of this document shows an example. If the report of assignment does not include numbered headings, the table of contents is not needed. Its purpose is to introduce the overall outline of the document and make it easier to read. Therefore, the table of contents has different heading levels to indicate heading hierarchy. It shows the heading words together with the number of the page where the text under each heading starts. The numbering and headings in the table of contents are identical to the ones used in the main body of the text. If the table of contents is generated with the wordprocessor’s function, this tool automatically updates the changes made in the text into the table of contents. Appendix 1 provides instructions for using these update functions. The table of contents also includes the heading REFERENCES, and the headings LIST OF FIGURES or LIST OF TABLES and APPENDICES, if necessary, but they are not numbered as chapters. In addition, the page number for the table of contents is not displayed on the page.

Uncommon characters and symbols, terms, self-made symbols and abbreviations can be listed separately. This list of definitions is placed after the table of contents and before the introduction without a page number. Its heading does not have a number, but it is listed in the table of contents. Standard symbols and abbreviations and common scientific terminology do not require definitions. Characters, terms, symbols and abbreviations can also be explained in the text. In this case, no separate list of definitions is necessary.


\textbf{Font and font sizes:}

\begin{itemize}
	\item{headings and body text with Arial, font size 12}
	\item{captions for figures and tables with Arial, font size 10}
\end{itemize}

\textbf{Margins}

\begin{itemize}
	\item{right margin 2 cm}
	\item{left margin 4.3 cm}
	\item{top margin 2.25 cm}
	\item{bottom margin 1.25 cm}
	\item{body text with line spacing 1.5}
	\item{captions for figures/tables and appendices with line spacing 1}
\end{itemize}















\end{document}
